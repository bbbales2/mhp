\documentclass[review]{elsarticle}
%Gummi|065|=)
\date{}
\usepackage{graphicx}
\usepackage{colortbl}
\usepackage{lineno}
\usepackage{hyperref}
\usepackage{capt-of}
\journal{}

\bibliographystyle{elsarticle-num}
\graphicspath{ {images/} }

\begin{document}
	\begin{frontmatter}
		\title{Segmentation-Free Image Processing and Analysis of Precipitate Shapes in 2D and 3D}

		%% Group authors per affiliation.
		\author[mech]{Ben Bales}
		\ead{bbbales2@gmail.com}
		\author[mat]{Tresa Pollock}
		\author[mech]{Linda Petzold}
		
		\address[mech]{University of California, Department of Mechanical Engineering, Santa Barbara, CA 93106-5070}
		\address[mat]{Materials Department, Engineering II Building, 1355 University of California, Santa Barbara, Santa Barbara, CA 93106-5050}

		\begin{abstract}
		Segmentation based image analysis techniques are routinely employed for quantitative analysis of complex microstructures containing two or more phases.  The primary advantage of these approaches are that spatial information on the distribution of phases is retained, enabling subjective judgements of the quality of the segmentation and subsequent analysis process. The downside is that computing micrograph segmentations with data in morphologically complex microstructures gathered with error-prone detectors is challenging and, if no special care is taken, the artifacts of the segmentation will make any subsequent analysis and conclusions uncertain. In this paper we demonstrate, using a two phase nickel-base superalloy microstructure as a model system, a new approach to analysis of precipitate shapes using a segmentation-free approach based on the histogram of oriented gradients approach.  The benefits of this approach for analysis of microstructure in two and three dimensions are demonstrated. This work is driven by the histogram of oriented gradiends feature descriptor (HOG), a classic tool in image analysis \cite{gradtex, hog, girsh}.
		\end{abstract}

		\begin{keyword}
			microstructure analysis\sep rafting\sep feature descriptor\sep 
		\end{keyword}
	\end{frontmatter}

	\section{Introduction}
		The strong driving force for development of rigorous property models for structural materials motivates quantitative analysis of microstructure across a spectrum of alloy systems \cite{echlinlenthe}. Since most engineering materials are multiphase in character, it is usually essential to isolate individual phases for analysis of size, shape and/or distribution in order to input this information into property models. The process for quantifying microstructure typically involves collection of 2D or 3D data on a pixel by pixel basis, followed by a segmentation operation to isolate individual phases within the microstructure. Shape metrics such as volumes, surface areas, or statistical moments \cite{twoDM, threeDM}) of the resulting precipitates are used to quantify the analysis. These metrics are chosen in part because of their similarity with how quantitative microstructure analysis has been performed manually \cite{sluytman, underwood}. The conjecture is that if enough micrographs can be captured and enough precipitates can be characterized, the shape statistics will yield good feature descriptors that can then be used in whatever classification or regression tasks that need to be addressed.
		
		The examples in this paper are of nickel-based superalloys. For these alloys, it is desirable to develop heat treatment cycles to adjust precipitate shapes for optimization of mechanical properties \cite{sluytman}. A unique feature of this class of alloys is the tendency for the precipitates to undergo directional coarsening during the application of external stresses at elevated temperatures [ref], a process known as "rafting". In both cases, measuring the shape of the microstructural precipitates can be very useful.

		The problem with this is that the segmentations are rarely trivial. Especially across data sets, but even within datasets, it can be very difficult to parameterize a segmentation algorithm to produce consistent results. Because the segmentation parameterization can strongly influence the shape statistics and because producing high quality segmentation frequently requires extensive fine tuning of segmentation parameters, it is difficult to argue that the resultant shape statistics are unbiased (with regards to the segmentation). The artifacts an automated segmentation of a g - g' microstructure might produce depend on the imaging modality, but typically include:

		\begin{enumerate}
			\item \label{prob1} A large number of single pixel $\gamma$ or $\gamma'$ precipitates appear due to detector noise in the original image
			\item Individual precipitates are merged into one large precipitate because the original image does not have high enough resolution for them to be segmented without high level material-specific knowledge
		\end{enumerate}
	
		\begin{figure}[!ht]
	  		\centering
			\includegraphics[width=0.5\textwidth]{renen4}
	  		\caption{ This is an example backscatter electron micrograph. $A$, $B$, and $B'$ all highlight typically difficult regions to segment. $A$ shows four precipitates that have been incorrectly merged by the segmentation algorithm. $B$ and $B'$ highlight areas where maybe there is a precipitate and maybe there isn't. In both cases the segmentation algorithm must make the decision between these two extremes. The choice is not obvious, and segmentation algorithms struggle. }
	  		\label{figure1}
		\end{figure}

		Figure \ref{figure1} shows an example image with the first two types of defects. These types of issues are not unique to superalloy microstructures, and techniques can be developed to address them \cite{comer, marc1, marc2, marc3}, though it is still very difficult to make them robust in any way, particularly across data sets. The simpler solution, if the information needed and the associated analysis allows, is to employ image analysis approaches that do not rely on segmentation.
	
		% Segmentation techniques are merely a holdover from how image analysis was done before computers.

		The goal of this paper is to highlight how a tool from computer vision, the Histogram of Oriented Gradients (HOG) feature detector, can be used to solve a wide variety of relevant classification and measurement problems robustly to the problems enumerated above. HOG feature detectors have a long, rich history of application in computer vision \cite{gradtex, hog, girsh}, but to the best of our knowledge have not been used in the study of microstructure.

		This paper is organized as follows. In Section \ref{relatedwork} we describe related work and in Section \ref{methodsandmaterials} we outline computation of a HOG feature descriptor. Finally in Section \ref{hog} we will demonstrate the effectiveness of the HOG feature descriptor a number of relevent microstructure characterization problems, where microstructural information is available in both 2D and 3D.
	
	\section{Related Work}\label{relatedwork}
		The classic segmentation-based text on microstructure analysis is Underwood\cite{underwood}. There are newer segmentation-free microstructure characterization techniques that use N-point statistics \cite{kalidindi1, kalidindi2} and sparse keypoint detectors like the Scale-Invariant Feature Transform (SIFT) \cite{decost}.

		Why, given the N-point statistics and SIFT-like techniques, develop something new? Basically, while a quantitative, segmentation-based feature vector is desired, it is also desirable that the feature vectors be easily interpretable by the lab scientists. The strength of a segmentation is that the data it produces (the outlines of the precipitates) is easy to directly interpret and understand. Our goal has been to develop a technique that makes both of these scenarios possible: quantitative analysis similar to the N-point statistics and SIFT features, and qualitative analysis similar to that done with segmentations. HOG descriptors, as used here, fill that gap. Their computation, interpretation, and analysis are straightforward.
	
		%The goal of this work was to avoid the classic segmentations in quantitative microstructure analysis. 
	
		 %our goal has been to identify a technique that 
	
		 %N-point and SIFT descriptors both contain more information than the HOG descriptors used here, but the downside to their information density is their size. It's difficult to work directly and intuitively with the numbers they produce.
	
		%Our goal has been to identify and technique for microstructure analysis that works qualitatively and quantitatively. Segmentation analysis fails on the quantitative part, and SIFT-like descriptors fail on the qualitative part. N-point statistics can be interpretted directly, but they have the nasty habit of making large data larger. Their direct human interpretation is limited, even when this data explosion can be handled by a computer.

	\section{Methods}\label{methodsandmaterials}
		Computation of the histogram of oriented gradients feature descriptor itself is straight forward. This technique is suitable for either standard scanning or tunneling electron microscope micrographs. The HOG feature detector is relatively coarse, and so the micrograph only needs to be of modest size, practically somewhere between 200x200 and 1000x1000 pixels. First, an approximate gradient at every point in the image is computed. This is most easily done by applying a light gaussian blur (just a couple pixel radius) to the image and taking finite differences. The Gaussian kernel should be large enough to remove the largest detector noise, but not enough to blur any important features. Finally, the values of the gradient are summed into a histogram of gradient angles weighted by gradient magnitudes.

		For image $F$ with Gaussian kernel $G$, the gradient at each point, $f_{ij}$, is given by $\nabla \left( G \ast F \right)_{ij}$. $f_{ij}$ is a vector with magnitude $\left| f_{ij} \right|$ and angle $\angle f_{ij}$. To build the histogram over angles, if each bin center is denoted as $\theta_k$ with radius $\delta$, then the value of the histogram $W$ at that bin center is given by

		\begin{equation}
			W \left( \theta_k \right) = \sum_{\left| \angle f_{ij} - \theta_k \right| < \delta} \left| f_{ij} \right|
		\end{equation}

	\section{The HOG Feature Detector}\label{hog}
	\subsection{Comparison of Microstructures}
	As stated before, a valuable feature of HOGs is the relative ease of their computation as compared to segmentations. Figure \ref{figure2} shows the comparison of two superalloy microstructures from F\"ahrmann\cite{faehrmann}.

	\begin{figure}[!ht]
  		\begin{center}
			\includegraphics[width=0.85\textwidth]{mollyhog}
	  		\caption{ Plot of the HOG feature descriptors (on the left) of transmission electron microscope micrographs of two superalloy samples \cite{faehrmann}. As can be seen, the precipitates in the top micrograph are more square than the precipitates in the bottom micrograph. }
	  		\label{figure2}
	  		
			\begin{tabular}{ r | >{\columncolor[gray]{0.5}}c | c | c | c | >{\columncolor[gray]{0.8}}c | c | c | c | >{\columncolor[gray]{0.8}}c }% | c
				\multicolumn{10}{c}{$\mathbf{\left| FFT \right|}$ \textbf{of HOG feature descriptors for Figure \ref{figure2}}} \\
				\hline
				Square (top) & 59 & 0.46 & 0.70 & 0.65 & 13 & 1.3 & 0.70 & 0.19 & 2.7 \\ \hline%& 0.85 
				Circle (bottom) & 88 & 1.4 & 2.9 & 0.29 & 2.2 & 0.32 & 0.29 & 0.57 & 0.18 \\ \hline%& 0.074 
				Index & 0 & 1 & 2 & 3 & 4 & 5 & 6 & 7 & 8 \\%& 9 
				\hline
		  	\end{tabular}
		  	\captionof{table}{ Magnitudes of the energies in the bins of the normalized HOG feature vector. As can be seen, the circular microstructure has more energy allocated in its zero bin (highlighted in dark grey), and the square microstructure has more energy in the fourth and eighth bin (highlighted in light grey). }
		  	\label{table1}
	  	
			\begin{tabular}{ r | c | c | c }
				\multicolumn{4}{c}{\textbf{HOG Scores for Figure \ref{figure2}}} \\
				\hline
				& \shortstack{Circle \\ (0Hz signal)} & \shortstack{Square \\ (4Hz harmonics)} & \shortstack{Layering \\ (2Hz harmonics)} \\
				\hline
				Square (top) & 20 & 0.95 & 0.97 \\
				Circle (bottom) & 455 & 0.29 & 0.80 \\
				\hline
			\end{tabular}
	  		\captionof{table}{ These are the HOG scores for Figure \ref{figure2}. As can be seen, the Circle score is much higher for the circular microstructure, and the Square score is much higher for the square microstructure. }
	  		\label{table5}
	  	\end{center}
	\end{figure}

	As shown in Figure \ref{figure2}, the HOG feature descriptor has peaks pointing in the normal directions of the facets in the top sample of Figure \ref{figure2}. This is because the histogram accumulates the magnitudes of gradients, so that where the gradient is large, large values are accumulated. In superalloy micrographs, the gradients are large at the edges of precipitates. The precipitates in the bottom sample are more spherical, and the HOG feature descriptor reflects this.
	
	The HOG feature descriptor does not reveal everything about a microstructure. For instance, if the cuboidal precipitates in the top sample of Figure \ref{figure2} were not globally aligned with each other, then the HOG feature descriptor for that would appear more uniform like that of the bottom sample. This could happen, for example, if the microstructure is from a polycrystalline sample.
	
	The HOG feature descriptor also doesn't directly reveal information about scale. For instance, it doesn't say that precipitates in the top sample are on average larger or smaller than the ones in the bottom sample.
	
	Taking a step back, even though the microstructures in Figure \ref{figure2} appear to be simple squares and circles, the microstructure samples themselves are 3D objects. It is possible, for instance, that if the top sample was cut on an angle the HOG plot would have more or less peaks due to symmetry and the sectioning plane.

	Thus far we have discussed the merits of the HOG as a qualitative feature descriptor. The simplest way to use it as a quantitative descriptor is to look at the magnitude of the FFT of the HOG feature descriptor and compare the relative amount of energy in different harmonics of the microstructure. The first ten bins of the absolute value of the FFTs of the HOG feature descriptors from Figure \ref{figure2} are shown in Table \ref{table1}.
	
  	We can compare how circular the two microstructures are by comparing the amount of energy in bin zero of the magnitude of the FFT to all of the other non-zero bins, and we can compare how square the two microstructures are by comparing the energy in every fourth non-zero frequency bin (four and eight highlighted in cyan) to the energy in every other non-zero frequency bin. A similar calculation can be done for 2Hz energies and all the harmonics. This can detect rafting in microstructures. These three numbers are a quick way to distill microstructure information and compare images in a rotation invariant way. The calculations for the materials from Figure \ref{figure2} are shown in Table \ref{table5}.
	
	\subsection{Detection of Rafting}
	The merits of HOG feature descriptors are easily demonstrated in the context of nickel-based superalloys for rafting, a unique tendency in this class of alloys for the precipitates to directionally coarsen during application of stresses at elevated temperatures. Figure \ref{figure3} shows two samples of Rene N5, one unrafted (top) and one rafted (bottom) along with plots of their HOG feature vectors. Table \ref{table2} shows the FFT-based scores for the rafted microstructure. The biggest change in score from the top (unrafted) microstructure to the bottom (rafted) one is the amount of energy in 4hz harmonics. The top microstructure has a large fraction of energy there, and the bottom microstructure has basically none.
	
	\begin{figure}[!ht]
		\begin{center}
			\includegraphics[width=0.85\textwidth]{renehog}
	  		\caption{ Plot of the HOG feature descriptors of BSE micrographs of a microstructure before (top) and after (bottom) rafting. The rafting is very clear in the HOG feature vector plots. }
	  		\label{figure3}
  		
			\begin{tabular}{ r | c | c | c }
				\multicolumn{4}{c}{Simple HOG Scores for Figure \ref{figure3}} \\
				\hline
				& \shortstack{Circle \\ (0Hz signal)} & \shortstack{Square \\ (4Hz harmonics)} & \shortstack{Layering \\ (2Hz harmonics)} \\
				\hline
				Base (top) & 4.4 & 0.69 & 0.91 \\
				Rafted (bottom) & 9.4 & 0.044 & 0.98 \\
				\hline
			\end{tabular}
			
	  		\captionof{table}{ These are the simple HOG scores for Figure \ref{figure3}. In these two samples, the biggest difference is that the Square score is much higher for the unrafted sample. For the rafted sample, the Square score is lowered but the Layering score remains high. }
	  		\label{table2}
		\end{center}
	\end{figure}
	
	\subsection{Analysis of 3D Transformations}
	HOG descriptors easily transfer to 3D datasets as well. While not as extensively deployed as their 2D counterparts, these feature detectors have found practical use in video datasets for action recognition (two spatial dimensions and one time dimension) \cite{hog3d1}. Again, in this application, they enable microstructure analysis without segmentation.

	Segmentation in 3D datasets can be quite difficult. All of the same problems with 2D data remain, except now visually verifying segmentations is more involved (requiring either volume rendering or a careful use of contour plots).

	A 3D HOG is simply a histogram across two dimensions. This can be tricky due to the difficulty imposed by gridding the surface of a spherical object, but it is possible. Figure \ref{figure4} shows a 3D microstructure along with its (adjusted) Histogram of Oriented Gradients plot. For visualization, it is usually desirable to adjust the values in the histogram to account for some bins covering a larger area on the sphere than others so that the values in the histogram are given per-area rather than just as a total sum. As shown in \ref{figure4}, there are six strong peaks corresponding to the 6 faces of the cube-shaped precipitates. Importantly, unlike the 2-D analysis, which would return a different shape based on the sectioning plane, the 3-D analysis would identify a cube shape regardless of sectioning plane. The clarity of the 3D HOG plot in Figure \ref{figure4} demonstrates the robustness of these feature detectors to noise.
	
	\begin{figure}[!ht]
  		\centering
    	\includegraphics[width=1.0\textwidth]{3dhog}
  		\caption{ On the left is the 3D HOG descriptor for a 3D BSE dataset (volume rendered on the right) of Rene N4 dataset collected with the Tribeam system \cite{tribeam}. It is the full 3D dataset associated with the image in Figure \ref{figure1}, which provides evidence that the HOG descriptor produces easily interpretable results even in the face of large amounts of noise (considerable effort was made to smooth the dataset for the volume rendering). }
  		\label{figure4}
	\end{figure}

	In analogy to the FFTs, it is possible to use rotation invariant spherical harmonics \cite{spherical} as feature vectors for analysis of 3D microstructures. Figure \ref{figure8} along with Table \ref{table3} show the results on a simulated coarsening experiment done by Wang \cite{ywang2}. The first descriptor in Table \ref{table3} (``Cube'') comes from looking at every fourth non-zero frequency bin of the rotation invarient spherical harmonics, and second descriptor (``Sphere'') comes from looking at the energy in the zeroth bin compared to everything else. The ``Cube`` score remains relatively stable compared to the ``Sphere`` descriptor which drops precipitously. This can be explained by thinking about the precipitate edge curvature remaining constant while the edge lengths increase. The precipitates are cuboids through the whole process, they are just becoming less and less spherical as the microstructure coarsens.
	
	\begin{figure}[!ht]
    	\begin{center}
			\includegraphics[width=0.85\textwidth]{coarsening}
	  		\caption{ $A$ is the base microstructure, $B$ is timestep two in the rafting process, and $C$ is timestep eight. The data is from Wang \cite{ywang2}. }
	  		\label{figure8}
  		
			\begin{tabular}{ c | c | c }
				\multicolumn{3}{c}{Coarsening experiment} \\
				\hline
				Timestep & Cube & Sphere \\
				\hline
				1 & 0.705 & 21.2 \\
				2 & 0.848 & 7.36 \\
				3 & 0.871 & 4.02 \\
				4 & 0.865 & 2.71 \\
				5 & 0.855 & 2.02 \\
				6 & 0.840 & 1.59 \\
				7 & 0.831 & 1.33 \\
				8 & 0.823 & 1.14 \\
				9 & 0.816 & 1.01 \\
				10 & 0.812 & 0.907 \\
				\hline
			\end{tabular}
			\captionof{table}{ The Cube and Sphere scores for these microstructures are computed similarly to the Circle and Square scores from Table \ref{table5} and Table \ref{table2}. }
			\label{table3}
		\end{center}
	\end{figure}
	
	Another simpler way to quantify 3D microstructure is to look at the mass moments of inertia of the HOG feature descriptor itself (computed as if the HOG were a thin-shelled spherical object with mass given by the value at each histogram point).
	
	\begin{figure}[!ht]
		\begin{center}
			\includegraphics[width=0.85\textwidth]{rafting}
	  		\caption{ $A$ is the base microstructure, $B$ is the result of rafting to columns, and $C$ is the result of rafting to layers. The data is from Wang \cite{ywang2}. }
	  		\label{figure5}
	  		
			\begin{tabular}{ c | c | c | c || c | c | c }
				\multicolumn{7}{c}{Moments for HOGs of rafted microstructures} \\
				\hline
				& \multicolumn{3}{c ||}{Columnar rafting} & \multicolumn{3}{ c}{Layered-by-layer rafting} \\
				\hline
				Time & $m_1$ & $m_2$ & $m_3$ & $m_1$ & $m_2$ & $m_3$ \\
				\hline
				1 & 0.641 & 0.644 & 0.715 & 0.613 & 0.692 & 0.694 \\
				2 & 0.626 & 0.629 & 0.744 & 0.569 & 0.715 & 0.717 \\
				3 & 0.614 & 0.618 & 0.768 & 0.529 & 0.735 & 0.736 \\
				... & & & & & & \\
				t & 0.556 & 0.559 & 0.885 & 0.169 & 0.915 & 0.916 \\
				t + 1 & 0.548 & 0.551 & 0.901 & 0.155 & 0.922 & 0.923 \\
				t + 2 & 0.544 & 0.546 & 0.911 & 0.146 & 0.926 & 0.928 \\
				t + 3 & 0.541 & 0.542 & 0.917 & 0.140 & 0.930 & 0.931 \\
				t + 4 & 0.539 & 0.540 & 0.921 & 0.135 & 0.932 & 0.933 \\
				t + 5 & 0.537 & 0.538 & 0.924 & 0.131 & 0.934 & 0.935 \\
				\hline
			\end{tabular}
			\captionof{table}{ Moments of the 3D HOG feature descriptor treated as a thin-shell object with mass given by the value of the HOG. At each timestep, the scale of the moments are renormalized. In the columnar rafting experiment, the moments slowly transform from all being equal to two smaller moments ($m_1$ and $m_2$) and one large one ($m_3$). In the layered rafting experiment, the moments slowly transform from being similar to one smaller moment ($m_1$) and two larger ones ($m_2$ and $m_3$). }
			\label{table6}
		\end{center}
	\end{figure}
  	
	For a cubic microstructure, there are six peaks in the HOG feature descriptor and three equivalent primary axis of rotation in the spherical HOG object. For a microstructure rafted into a columnar structure, there are only four strong peaks in the HOG feature detector, and likewise two equivalent axes of rotation with large moments of inertia and a third with a smaller moment. For a microstructure rafted into a layer by layer structure, the HOG feature detector has only two strong peaks and there is a single large moment of inertia and two smaller ones for the spherical HOG object. Figure \ref{figure5} shows volume renderings of these two types of rafting that come from simulations done by Wang \cite{ywang2}. Table \ref{table6} shows the moment analysis of these experiments which reflects the behavior described above (data also from Wang \cite{ywang2}).
	
	\section{Conclusion}
	
	This paper demonstrates that in many types of basic microstructure analysis it is possible to substitute an easy to compute histogram of oriented gradients feature vector in place of difficult to compute segmentation statistics. While the HOG has limitations, it is easy to compute and is more robust to common noise sources in superalloy microscopy techniques and can be applied in a number of interesting applications in both 2D and 3D datasets.
	
	\section{Acknowledgements}
	
	%The SEM images in \ref{figure2} in this paper come from F\"ahrmann \cite{faehrmann}. The 3D dataset in Figure \ref{figure4} was collected by Will Lenthe and McLean Echlin on the Tribeam \cite{tribeam}. The simulated data in Figures \ref{figure8} and Figure \ref{figure5} comes from Wang \cite{ywang2}. 
	The BSE micrographs in \ref{figure3} come from data collected by Luke Rettenberg. This material is based upon work supported by the National Science Foundation under Grants No. DMR 1233704 and DMR 1534264.

	\bibliography{bibliography}
\end{document}
